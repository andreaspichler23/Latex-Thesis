\chapter{Physics of the VIP2 experiment}
\label{chap:Physics}

%discovery of the pep?; spin, pep and the spin statistics connection - 2nd quantization - symm/antisymm wfkt; exchange interaction; symmetrization postulate??; group theory + young diagrams; theories of violation of spin Statistics, mg superselection rule, electrons in a conductor - formation of electric current - decoherence of electrons ? - interaction of an electron with an atom - cascading process; current on/off -> vip2 mesurement principle; and why it is better than other experiments (steady state interactions) 

\section{Physics Basics}
\label{sec:PhysicsBasics}

\subsection{The Pauli Exclusion Principle}
\label{sec:pep}

To explain the spectra of alkali atoms recorded with a magnetic field (Zeemann effect), Wolfgang Pauli postulated a $4^{th}$ quantum number for electrons in the early 1920s. The new quantum number was an addition to the quantum numbers already know to that time, which are nowadays called principal quantum number \textit{n}, angular momentum quantum number \textit{l} and magnetic quantum number \textit{$m_{l}$}. He named it a ``two-valuedness not describable classically'' \cite{Pauli1946}. This $4^{th}$ quantum number was later called the electron's spin. Another problem he was working on at that time was the series of integer numbers 2, 8, 18, 32, etc., which was determining the lengths of the lines in what we call the periodic table of elements. It was furthermore known to him that the number of electronic energy levels in an alkali atom were the same as the number of electrons in the closed shell of the rare gas with the same principal quantum number. He used this information to formulate the Pauli Exclusion Principle: The number of electrons in closed subgroups can be reduced to one, if the division of the groups (by giving them values of the four quantum numbers) is carried so far that every degeneracy is removed. An entirely non-degenerate level is closed, if it is occupied by a single electron \cite{Pauli1946}. This is equivalent to saying that every state corresponding to a set of quantum numbers \textit{n}, \textit{l}, \textit{$m_{l}$} and \textit{$m_{s}$} can only be occupied by one electron. Wolfgang Pauli won the Nobel Prize in physics for the formulation of the Pauli Exclusion principle in 1945. It was first formulated for electrons, but later on extended to all fermions.

\subsection{Quantum Mechanical Angular Momenta}
\label{sec:AngMom}

%orbital angluar mom + spin: 2 diff types , spin angular momenta can take half integer values; spin has magnitude: all particles of same type- same, kg m2s-1, dimensionless spin quantum number, spin can not be made higher or lower (contrast to ang mom), algabra and formulas
%angular momentum + magnetic moment -classically; lz, l2 operator,eigenvalues, degeneracy; commutation relations -> also with hamiltonian -> conservation

From a classical point of view an angular momentum is defined as: $\vec{L} = \vec{r} \times \vec{p}$. In this formula $\vec{L}$ is the angular momentum, $\vec{r}$ is the vector to the particle from the origin and $\vec{p}$ is the momentum of the particle. The magnetic moment of a particle with charge q moving in a circle with radius \textit{r} is defined as $\vec{\mu} = I \times \vec{A}$. Here $\vec{\mu}$ is the magnetic moment, $\vec{A} = r^{2}\pi * \vec{e_{A}}$ is the area that the particle's movement is encircling and \textit{I} is the current. The current the particle generates can be written as:
\begin{equation}
 I = \frac{q}{T} = \frac{q}{\frac{2r\pi}{v}} = \frac{qv}{2r\pi}
\end{equation} 
Here, \textit{v} is the particle's velocity and \textit{T} is the time it needs for one full circle. After dropping the vector arrows due to orthogonality, the magnetic moment can be written as:
\begin{equation}
 \mu = IA = \frac{qv}{2r\pi}r^{2}\pi = \frac{qvr}{2} = \frac{q}{2m} rvm = \frac{q}{2m}L
 \label{eq:magnMoment}
\end{equation} 
Where $\frac{q}{2m} = \frac{\mu}{L}$ is called the gyromagnetic ratio. For electrons formula \ref{eq:magnMoment} is often rewritten in the form
\begin{equation}
 \mu_{e} = g\mu_{B}\frac{L}{\hbar} \hspace{2em} \mu_{B} = \frac{e\hbar}{2m_{e}} \hspace{2em} \hbar = 1.054571800(13)\times10^{-34} [\frac{m^{2}kg}{rad\times s}]
 %\hbar = 10^{−34} \frac{m^{2}kg}{s}
 \label{eq:mue}
\end{equation} 
where $\mu_{e}$ is the magnetic moment of the electron, $m_{e}$ is the electron mass and $\mu_{B}$ is the Bohr magneton. The Bohr magneton is the expected ratio between the magnetic moment $\mu_{e}$ and the dimensionless value $\frac{L}{\hbar}$. The \textit{g-factor} parametrizes deviations from the expected value \textit{g} = 1, which could arise if for example the charge density distribution is different from the mass density distribution.

In analogy to classical mechanics, the angular momentum can be written in quantum mechanics as a cross product of the position operator \textit{$\widehat{x}$} and the momentum operator \textit{$\widehat{p}$}: $\widehat{L} = \widehat{x} \times \widehat{p}$. In position basis this can be written as
\begin{equation}
 \widehat{L} = \widehat{x} \times \widehat{p} = \frac{\hbar}{\imath}(\vec{x}\times \vec{\nabla}) \hspace{2em} \vec{\nabla}= \left (
  \begin{tabular}{c}
  $\frac{\partial}{\partial x}$\vspace{0.2em}\\
  $\frac{\partial}{\partial y}$\vspace{0.2em}\\
  $\frac{\partial}{\partial z}$\vspace{0.2em}
  \end{tabular}
\right )
\end{equation} 
In index notation, this operator can be written as follows: $\widehat{L_{i}}=\epsilon_{ijk} \widehat{x_{j}}\widehat{p_{k}}$ with $\epsilon_{ijk}$ being the antisymmetric Levi-Civita tensor. The indices $i,j,k$ correspond to the 3 spatial dimensions. $\widehat{L_{i}}$ and $\widehat{L_{j}}$ do not commute as 
\begin{equation}  
\label{eq:LCommutator1}
 [\widehat{L_{i}},\widehat{L_{j}}]=\imath\hbar\epsilon_{ijk}\widehat{L_{k}} \hspace{4em}  [\widehat{X},\widehat{Y}] = \widehat{X}\widehat{Y} - \widehat{Y}\widehat{X}                                                                                                                                                                                                \end{equation} 
Here $[\widehat{X},\widehat{Y}]$ denotes the commutator of $\widehat{X}$ and $\widehat{Y}$. For readability, the hats of operators will be omitted from now on. The entries $L_{i}$ of the angular momentum operator commute with the rotation invariant form $L^{2} = L_{x}^2+L_{y}^2+L_{z}^2$. This is the operator of the squared norm of the angular momentum. $L_{i}$ and $L^{2}$ are compatible variables and can be measured simultaneously. This also means that
\begin{equation}
\label{eq:LCommutator2}
 [L_{i},L^{2}]=0
\end{equation} 
For any given system, the following relations for the eigenvalues of these operators hold:
\begin{equation}
\label{eq:ml}
 L_{i}\ket{\phi}=m_{l}\hbar\ket{\phi} \hspace{1cm} m_{l}\in{...-2,-1,0,1,2,...}
\end{equation} 
\begin{equation}
\label{eq:l}
 L^{2}\ket{\phi}=\hbar^{2}l(l+1)\ket{\phi} \hspace{1.4cm} l\in{0,1,2,...} \hspace{2em}
\end{equation} 
with \textbar\textit{$m_{l}$}\textbar $\leq$ \textit{l}. Considering particles without spin and a Hamiltonian symmetric under rotations of the form $H=\frac{p^{2}}{2m}+V(r)$ (like the atom), the angular momentum is conserved and commutes with the Hamiltonian
\begin{equation}
\label{eq:commutatorL3}
 [H,L_{i}]=[H,L^{2}]=[L_{i},L^{2}]=0
\end{equation} 
Therefore $L_{i}$ and $L^{2}$ are conserved quantities. For systems without spin, $H$, $L^{2}$ and $L_{i}$ form a complete set of commuting observables with the corresponding quantum numbers $n$ (principal quantum number), $l$ (angular momentum quantum number) and $m_{l}$ (magnetic quantum number). Introducing the rotation invariant Coulomb potential for V(r)
\begin{equation}
 V(r)=-\frac{1}{4\pi\epsilon_{0}}\frac{q}{r}
\end{equation} 
one finds that angular momentum quantum number $l$ always needs to be smaller than the principal quantum number $n$ ($l<n$). Only considering pure Coulomb interaction, the eigenstates with principal quantum number $n$ belonging to the eigenvalue $E_{n}$ are $n^{2}$-fold degenerate.
 
\subsection{The Spin}
\label{sec:Spin}

%orbital angluar mom + spin: 2 diff types , spin angular momenta can take half integer values; spin has magnitude: all particles of same type- same, kg m2s-1, dimensionless spin quantum number, spin can not be made higher or lower (contrast to ang mom), algabra and formulas; J=L+S commutation with H; pauli matrices

The spin is an intrinsic form of angular momentum carried by elementary particles. It has a definitive and non-modifiable magnitude for each particle type. This is a difference to the section \ref{sec:AngMom}, where the angular momenta, described by the quantum number $l$, could change in magnitude. Wolfgang Pauli was the first to propose the concept of spin in 1925. Analogous to the relation between angular momentum and magnetic moment \ref{eq:mue}, the relation between spin and angular momentum can be written as
\begin{equation}
 \mu_{s} = g\mu_{B}\frac{S}{\hbar} \hspace{2em} \mu_{B} = \frac{e\hbar}{2m}
\end{equation} 
where $\mu_{s}$ is the magnetic moment of a particle due to its spin and $S$ is the magnitude of this spin. Unlike for angular momenta described in \ref{sec:AngMom}, the value for $g$ $\neq$ 1. From the Dirac equation, a value of $g$ = 2 can be obtained. Corrections for example from Quantum Electrodynamics further alter this value on the \%-level. 

Analogous to equation \ref{eq:LCommutator1} and \ref{eq:LCommutator2}, commutation relations can be found for the spin:
\begin{equation}
 [S_{i},S_{j}]=\imath\hbar\epsilon_{ijk}S_{k}
\end{equation} 
\begin{equation}
 [S_{i},S^{2}]=0
\end{equation} 
Here $S_{i}$ are the spin components in one spatial direction and $S^{2}$ is the squared norm of the spin. Furthermore, analogous to equations \ref{eq:ml} and \ref{eq:l} following relations hold for the spin:
\begin{equation}
 S^{2}\ket{\phi}=\hbar^{2}s(s+1)\ket{\phi} \hspace{2em} s\in{0,\frac{1}{2},1,...} 
\end{equation} 
\begin{equation}
 S_{i}\ket{\phi}=m_{s}\hbar\ket{\phi} \hspace{2em} \mid m_{s} \mid \leq \textit{s}
\end{equation} 
Here $s$ is the spin quantum number and $m_{s}$ is the spin projection quantum number. The big difference between $s$ and its analogon $l$ from the previous chapter is that $s$ can also take half-integer values. Another difference is, that $s$, unlike $l$, cannot be changed and is intrinsic for each particle type. Particles with half-integer spin quantum number are called \textit{fermions}, particles with integer spin are called \textit{bosons}. In particular, leptons such as electrons have $s$ = $\frac{1}{2}$ and therefore $m_{s}$ can take the values $\pm$ $\frac{1}{2}$. For a system containing particles with spin, $n$, $l$, $m_{l}$ and $m_{s}$ form a complete set of commuting observables. The total angular momentum ($\vec{J}$) can be defined as the sum of the orbital angular momentum and spin $\vec{J}$ = $\vec{L}$ + $\vec{S}$. Analogous expressions to equations \ref{eq:LCommutator1} - \ref{eq:commutatorL3} hold for the total angular momentum $\vec{J}$ in systems containing particles with spin.

Considering a system containing a particle with $s$ = $\frac{1}{2}$. The basis in which $S^{2}$ and $S_{z}$ are diagonal consists of two states $\ket{s,m_{s}}$ = $\ket{\frac{1}{2},\pm \frac{1}{2}}$. These states can be identified with the basis vectors $ e_{1} = \begin{pmatrix}1 \\ 0\end{pmatrix}$ and $ e_{2} = \begin{pmatrix}0 \\ 1\end{pmatrix}$ and are often referred to as ``spin-up'' and ``spin-down'' relative to a defined z-direction. The action of operators on these states is as follows:
%
\begin{equation}
 S^{2}\ket{\frac{1}{2},\pm \frac{1}{2}} = \hbar^{2}\frac{1}{2}(\frac{1}{2}+1) \ket{\frac{1}{2},\pm \frac{1}{2}}
\end{equation} 
%
\begin{equation}
 S_{z}\ket{\frac{1}{2}, \frac{1}{2}} = \hbar \frac{1}{2} \ket{\frac{1}{2}, \frac{1}{2}}
\end{equation} 
%
\begin{equation}
 S_{z}\ket{\frac{1}{2}, -\frac{1}{2}} = -\hbar \frac{1}{2} \ket{\frac{1}{2}, -\frac{1}{2}}
\end{equation} 
%
\begin{equation}
\label{eq:ladder1}
 S_{+}\ket{\frac{1}{2}, \frac{1}{2}} = 0 \hspace{2em} S_{-}\ket{\frac{1}{2}, \frac{1}{2}} = \hbar \ket{\frac{1}{2}, -\frac{1}{2}}
\end{equation} 
%
\begin{equation}
\label{eq:ladder2}
 S_{+}\ket{\frac{1}{2}, -\frac{1}{2}} = \hbar \ket{\frac{1}{2}, \frac{1}{2}} \hspace{2em} S_{-}\ket{\frac{1}{2}, -\frac{1}{2}} = 0
\end{equation} 
%
The \textit{ladder operators} ($S_{\pm}$) were used in equations \ref{eq:ladder1} and \ref{eq:ladder2}. A ladder operator increases ($S_{+}$) or decreases ($S_{-}$) the azimuthal quantum number of a state and for an angular momentum $\vec{J}$ with quantum numbers $j$ and $m_{j}$ defined as
%
\begin{equation}
 J_{\pm} \ket{j, m_{j}} = \hbar \sqrt{(j\mp m_{j}) (j \pm m_{j} +1) } \ket{j,m_{j}\pm 1}
\end{equation} 
%
In the mentioned basis, the operators $S_{x}$, $S_{y}$ and $S_{z}$ can be written as
\begin{equation}
 S_{x}=\frac{\hbar}{2} \begin{pmatrix}0 & 1 \\ 1 & 0\end{pmatrix}, \hspace{2em} S_{y}=\frac{\hbar}{2} \begin{pmatrix}0 & -\imath \\ \imath & 0\end{pmatrix}, \hspace{2em}%
 S_{z}=\frac{\hbar}{2} \begin{pmatrix}1 & 0 \\ 0 & -1\end{pmatrix}
\end{equation} 
These are the so called \textit{Pauli matrices}.

\subsection{Indistinguishability, Symmetrization Postulate and Superselection Rule}
\label{sec:IndSymPostSS}

%indistinguishability of identical particles -> [P,A] = [P,H] =0 invariance of observables under permutations * -> invariance of the hamiltonian -> possible wfkt -> symmetrization postulate (arguments against) -> -> divisin into fermions and bosons -> simple argument for PEP; bosonic and fermionic statistics (occupation number)

The following section is loosely based on \cite{Sperandio2008}. Considering a state $\phi$ = $\phi(1,2,..,i,...,j,...N)$, where the variables 1,2,... denote the spatial and the spin degrees of freedom. The action of the permutation operator $P_{i,j}$ on this state is defined as:
\begin{equation}
 P_{i,j}\phi(1,2,..,i,...,j,...N) = \phi(1,2,..,j,...,i,...N)
\end{equation} 
The indistinguishability of identical particles implies that states that differ only by a permutation of identical particles can not be distinguished by any measurement. In quantum mechanics, a measurement is expressed as the expectation value of a hermitian \footnote{A hermitian or self-adjoint operator is an operator for which the relation $A^{\dag} = A$ holds. The hermitian conjugation $^{\dag}$ corresponds to transposition combined with complex conjugation. A hermitian operator has real eigenvalues and eigenvectors for different eigenvalues are orthogonal.} operator \textit{A}. This statement can be expressed for a state $\phi$ as the following equation:
\begin{equation}
 \braket{\phi|A|\phi} = \braket{\phi|P^{\dag}AP|\phi}
\end{equation} 
From this equation follows that the permutation operator commutes with every observable, as it holds for every state $\phi$ and it follows that $P^{\dag}AP$ = \textit{A}. Therefore \textit{PA} = \textit{AP} which implies commutation of the two operators. Specifically, the energy of a quantum mechanical system also must not depend on the permutation of identical particles. From above considerations it follows that:
\begin{equation}
 [P,H] = 0
\end{equation} 
where \textit{H} is the Hamiltonian. 

An infinitesimal time evolution of a state is given by the Schrödinger equation
\begin{equation}
 \partial_{t}\ket{\phi(t)} = \frac{1}{\imath\hbar} H \ket{\phi(t)} \Rightarrow \ket{\phi(t+\delta t)} = (1 + \frac{\delta t}{ \imath\hbar} H + O(\delta t^{2}))\ket{\phi(t)}
\end{equation} 
For a time-independent Hamiltonian \textit{H}, \textit{n}$\rightarrow \infty$ infinitesimal time steps between a start time $t_{0}$ and time $t$ give the time evolution operator $U(t-t_{0})$:
\begin{equation}
\label{eq:timeEvOp}
 \ket{\phi(t)} = U(t-t_{0})\ket{\phi(t_{0})} \hspace{2em} U(t-t_{0}) = e^{-\frac{\imath}{\hbar}(t-t_{0})H}
\end{equation} 
As the permutation operator commutes with the Hamiltonian, it also commutes with the time evolution operator $U$,
\begin{equation}
 [P,U] = 0
\end{equation} 
because of equation \ref{eq:timeEvOp}. Therefore, the permutation symmetry of a state is conserved. This is called the \textit{Messiah-Greenberg (MG) superselection rule}. It is important to note that above considerations are only viable for systems where the number of particles is conserved and that the symmetry of a system is not necessarily preserved in systems with a non-constant particle number (see for example \cite{Messiah1964}). 

Considering now a system of 2 particles, in which the state $\phi(1,2)$ is a solution of the Schrödinger equation
\begin{equation}
 H \ket{\phi(1,2)} = E \ket{\phi(1,2)}
\end{equation} 
As the Hamiltonian commutes with the permutation operator, $P_{12} \ket{\phi(1,2)} = \ket{\phi(2,1)}$ is also a solution to this equation with the same Hamiltonian $H$ and the same eigenvalue $E$. All linear combinations of these two functions are also solutions of the equation. The linear combinations $\ket{\Phi} = \ket{\phi(1,2)} \pm \ket{\phi(2,1)}$ represent solutions corresponding to positive and negative symmetry with respect to particle exchange. 

The situation is a bit more complex for a system with 3 particles. In case the state $\ket{\phi(1,2,3)}$ solves the Schrödinger equation, the linear combination:
\begin{equation}
 \ket{\phi(1,2,3)} + \ket{\phi(1,3,2)} - \ket{\phi(3,2,1)}
 \label{eq:3partState}
\end{equation} 
also solves the Schrödinger equation. For an exchange of particles 1 and 2, this state becomes
\begin{equation}
 \ket{\phi(2,1,3)} + \ket{\phi(2,3,1)} - \ket{\phi(3,1,2)}
\end{equation} 
The state is not an eigenstate of the permutation operator $P_{12}$, as it is not the same as the one given in \ref{eq:3partState}. Therefore not all solutions of the Schrödinger equation need to be eigenfunctions of the permutation operator. A special case are the linear combinations with negative
\begin{equation}
 \ket{\phi(1,2,3)} - \ket{\phi(1,3,2)} - \ket{\phi(2,1,3)} + \ket{\phi(2,3,1)} + \ket{\phi(3,1,2)} - \ket{\phi(3,2,1)}
\end{equation} 
and positive
\begin{equation}
 \ket{\phi(1,2,3)} + \ket{\phi(1,3,2)} + \ket{\phi(2,1,3)} + \ket{\phi(2,3,1)} + \ket{\phi(3,1,2)} + \ket{\phi(3,2,1)}
\end{equation} 
symmetry with respect to particle exchange. These linear combinations are called completely (anti-)symmetric. This means that an application of the permutation operator for any pair of particles gives a negative or positive sign for the state $P\ket{\Phi}=\pm\ket{\Phi}$.

For a general system of $N$ particles, the symmetry of different linear combinations of wave functions are described by \textit{Young diagrams} (see e.g. \cite{Kaplan2013}). A Young diagram represents an irreducible representation of the permutation group \footnote{The permutation group $S_{N}$ is a group whose elements are the permutations of a set with $N$ elements.}. An example for such diagrams is shown in figure \ref{fig:YoungDiag}.
\begin{figure}[h]
 \centering
 \includegraphics[width=0.6\textwidth]{./Figures/YoungDiagram.png}
 % YoungDiagram.png: 1085x347 pixel, 211dpi, 13.06x4.18 cm, bb=0 0 370 118
 \caption{Young diagrams of the $S_{3}$ permutation group for antisymmetric (left), mixed (middle) and symmetric (right) permutation symmetry.}
 \label{fig:YoungDiag}
\end{figure}
Each box of a Young diagram symbolizes a particle and the spatial relation of two boxes symbolizes the permutation symmetry of the state with respect to exchange of the particles in the boxes. Two boxes arranged vertically stand for antisymmetric exchange symmetry and two boxes aligned horizontally correspond to symmetric exchange symmetry. From the description of this kind it follows that all states described by a Young diagram are eigenstates of the permutation operator. In figure \ref{fig:YoungDiag} different exchange symmetries for systems with three particles are shown. It is important to note that there is not only the completely (anti-)symmetric exchange symmetry (left/right side), but also the state with a positive symmetry for the exchange of one pair of particles and negative symmetry for another pair. This state is called a mixed-symmetry state.

The \textit{symmetrization postulate} states that from the 3 different permutation symmetries in figure \ref{fig:YoungDiag}, only the left and the right ones are realised in nature \cite{Messiah1964}. Due to their form they are called the one-dimensional representation of the permutation group. The usual proof of this postulate is like this:

The indistinguishability of identical particles results in the fact that a permutation of two particles should only multiply the wave function only by an insignificant phase factor $e^{\imath\alpha}$ with $\alpha$ being a real constant.
\begin{equation}
 P_{12}\ket{\phi(1,2)} = \ket{\phi(2,1)} = e^{\imath\alpha}\ket{\phi(1,2)}
\end{equation} 
One more application of the permutation operator gives
\begin{equation}
\label{eq:phaseFactor}
 P_{12}P_{12}\ket{\phi(1,2)} =  \ket{\phi(1,2)} = e^{\imath\alpha}e^{\imath\alpha}\ket{\phi(1,2)} = e^{2\imath\alpha}\ket{\phi(1,2)}
\end{equation} 
or
\begin{equation}
\label{eq:phaseFactor2}
 e^{2\imath\alpha} = 1 \Rightarrow e^{\imath\alpha} = \pm 1
\end{equation} 
This proof is incorrect, as it is shown in \cite{Kaplan2013}. One argument against this proof is that the indistinguishability of identical particles only requires the squared norm of the wave function to be invariant under permutations:
\begin{equation}
 P_{12}\mid\ket{\phi(1,2)}\mid^{2} = \mid\ket{\phi(1,2)}\mid^{2}
\end{equation} 
For a function to satisfy this relation it is sufficient that it changes under permutations as:
\begin{equation}
 P_{12}\ket{\phi(1,2)} = e^{\imath\alpha(1,2)}\ket{\phi(1,2)}
\end{equation} 
where 1 and 2 are as in the above formulas the space and the spin coordinates of the two particles. Thus the phase factor can be a function of the permutation and of the coordinates. That means that in general, equations \ref{eq:phaseFactor} and \ref{eq:phaseFactor2} do not hold. Consequently the symmetrization postulate states the fact that only the one-dimensional representations of the permutation group meaning the fully (anti-) symmetric states have yet been observed in nature and the solution of the Schrödinger equation can belong to any representation of the permutation group, not only the one-dimensional ones.

\subsection{Fermions, Bosons and the Spin-Statistics Connection}

As mentioned in section \ref{sec:Spin}, different particles have different intrinsic spin, which cannot be altered. As was shown in 1940 by Wolfgan Pauli \cite{Pauli1940}, the spin of a particle determines which of the two possible representations of the permutation group it belongs to. 

Particles with integer spin (s = 0,1,2, ...) have symmetric wave functions with respect to particle exchange. These particles are called \textit{bosons} (after indian physicist Satyendra Nath Bose). Their corresponding Young diagram is of the type on the right side of figure \ref{fig:YoungDiag}. Elementary bosonic particles are for example the force carrier particles of strong, weak and electromagnetic interaction: the gluon, the W and Z bosons and the photon. Another example for an elementary boson is the Higgs particle. Composite bosons can be made up out of particle with half integer or with integer spin. Example for this kind of particles are mesons, which are made up out of two quarks with s = $\frac{1}{2}$. 

The occupation number of bosons follows the Bose-Einstein statistics. The expected number of particles in an energy state is in this case:
\begin{equation}
 N(E) = \frac{1}{e^{\frac{E-\mu}{kT}}-1}
\end{equation} 
where $E$ is the energy of the state, $\mu$ is the chemical potential, $k$ is the Boltzmann constant and $T$ is the absolute temperature. A consequence of this statistics is that more than one bosonic particles can occupy the same quantum state. The commutation relations of creation ($a^{\dag}$) an annihilation ($a$) operators for bosonic particles in states $X$ and $Y$ are:
\begin{equation}
\label{eq:bosComm}
 [a_{X},a_{Y}] = 0 \hspace{2em} [a^{\dag}_{X},a^{\dag}_{Y}] = 0 \hspace{2em} [a_{X},a^{\dag}_{Y}] = \delta_{X,Y}
\end{equation} 

Particles with half-integer spin (s = $\frac{1}{2},\frac{3}{2},...$) have antisymmetric wave functions with respect to particle exchange. This means changing the position of two particles multiplies the wave function with a minus sign. These particles are called \textit{fermions} (after the italian physicist Enrico Fermi). Their corresponding Young diagram is of the type on the left side of figure \ref{fig:YoungDiag}. Fermions can be elementary particles like quarks and leptons (e.g. electrons), but also composite particles like neutrons, protons or even atoms. 

The occupation number of fermions follows the Fermi-Dirac statistics. The expected number of particles in an energy state is in this case:
\begin{equation}
 N(E) = \frac{1}{e^{\frac{E-\mu}{kT}}+1}
\end{equation} 
As the exponential function is always positive, the expected occupation number is always smaller or equal to one. This means that every energy state can only be occupied by one fermion. This is known as the \textit{Pauli Exclusion Principle (PEP)} (see also section \ref{sec:pep}). It can be seen that particles with purely fermionic exchange symmetry cannot be in the same state from the  considerations of a system consisting of two fermionic particles with two possible states. The unnormalized antisymmetric wave function is:
\begin{equation}
 \ket{\Phi_{a}} = \ket{\phi(1,2)} - \ket{\phi(2,1)} 
\end{equation} 
The wave function for the 2 fermionic particles being in the same state is (in this case in state 1):
\begin{equation}
 \ket{\Phi_{a}} = \ket{\phi(1,1)} - \ket{\phi(1,1)} = 0
\end{equation} 
So the antisymmetric wave function of two particles being in the same state is equal to zero. Therefore two fermionic particles can not be in the same state.

The relation described above between a particle's spin and its statistics is called the \textit{Spin-Statistics Connection}. The anticommutation relations of creation ($a^{\dag}$) an annihilation ($a$) operators for fermionic particles in states $X$ and $Y$ are:
\begin{equation}
\label{eq:fermComm}
 \{a_{X},a_{Y}\} = 0 \hspace{2em} \{a^{\dag}_{X},a^{\dag}_{Y}\} = 0 \hspace{2em} \{a_{X},a^{\dag}_{Y}\} = \delta_{X,Y}
\end{equation} 
From \ref{eq:fermComm} it can be seen that adding or removing two particles from a state results in zero (as for example $\{a_{X},a_{Y}\} = a_{X}a_{Y} + a_{Y}a_{X} = 0$). Consequently not more than one particle can be in the same state for fermions. The action of these operators for fermionic particles on the unoccupied vacuum state $\ket{0}$ and the state occupied by one particle $\ket{1}$ can be written as
\begin{equation}
\label{eq:fermCreatAnnihOp}
 a \ket{0} = 0 \hspace{2em} a \ket{1} = \ket{0} \hspace{2em} a^{\dag} \ket{0} = \ket{1} \hspace{2em} a^{\dag} \ket{1} = 0
\end{equation} 

In the literature many proofs for the Spin-Statistic connection exist (e.g. \cite{Pauli1940}, \cite{Schwinger1958}). A clear set of assumptions for this proof was presented by Lüders and Zumino in \cite{Luders1958}. The authors present 5 postulates plus gauge invariance as a foundation of their proof:
\begin{itemize}
 \item Invariance with respect to the proper inhomogeneous Lorentz group (which contains translations, but no reflections)
 \item Locality - 2 operators of the same field separated by a spacelike interval either commute or anticommute
 \item The vacuum is the state of the lowest energy
 \item The metric of the Hilbert space is positive definite
 \item The vacuum is not identically annihilated by a field
\end{itemize}
It is worth noting that the mentioned proof also holds for interacting fields. Another interesting point is that the Spin-Statistics connection does not hold for two spatial dimensions. The concept of \textit{anyons}, a class of particles which does not follow bosonic or fermionic statistics, was presented in \cite{Stern2008} in this context.

\section{Theories of Violation of Spin-Statistcs}
\label{sec:thVioPEP}

There have been many attempts to find a theory of quantum mechanics which is consistent with a violation of Spin-Statistics. Some of the most important theories will be discussed in the following section.

\subsection{Parastatistics}

The first proper quantum statistical generalization of fermi and bose statistics was done by Green \cite{Green1953}, \cite{Greenberg2000}. He noticed that the commutator of the occupation number operator of the state $X$: $N_{X} = a^{\dag}_{X} a_{X} $ with the annihilation and creation operators is the same for fermions and bosons:
\begin{equation}
 [N_{X},a^{\dag}_{Y}] = \delta_{X,Y} \hspace{0.2em} a^{\dag}_{Y}
\end{equation} 
As a result, the number operator can be written as:
\begin{equation}
 N_{X} = \frac{1}{2}[a^{\dag}_{X},a_{X}]_{\pm} + const
\end{equation} 
The $\pm$ sign denotes the (anti-)commutator for the (bosonic) fermionic case. The expression for the transition operator $N_{X,Y}$, which annihilates a particle in state $Y$ and creates a particle in state $X$, leads to the trilinear commutation relation \footnote{A trilinear form is a function of 3 arguments, in which every argument enters only to first order.} for parabose and parafermi statistics:
\begin{equation}
\label{eq:triComm}
 [[a^{\dag}_{X},a_{Y}]_{\pm},a_{Z}^{\dag}] = 2 \hspace{0.2em} \delta_{Y,Z} \hspace{0.2em} a_{X}^{\dag}
\end{equation} 
These relations have an infinite set of solutions, each of them corresponding to an integer $p$. The integer $p$ is the order of the parastatistics and gives the number of particles that can be in an antisymmetric state in the case of parabosons and the number of particles that can be in a symmetric state in case of parafermions. The case $p$ = 1 corresponds to normal fermionic or bosonic statistics. It was shown that the squares of all norms are positive for states satisfying Green's trilinear commutation relation. Nevertheless, the violations introduced by these statistics is large and no precision experiments are needed to rule them out.

\subsection{The Ignatiev and Kuzmin Model and Parons}

In 1987, A. Ignatiev and V. Kuzmin constructed a model of one oscillator with three possible states: a vacuum state with no occupancy, a one particle state and with a small amplitude parametrized by a  $\beta$ a state occupied by two particles \cite{Ignatiev1987}. The creation and annihilation operators connect these three states (analogous to \ref{eq:fermCreatAnnihOp}) as:
\begin{equation}
 a \ket{0} = 0 \hspace{2em} a \ket{1} = \ket{0} \hspace{2em}  a \ket{2} = \beta \ket{1}
\end{equation} 
\begin{equation}
 a^{\dag} \ket{0} = \ket{1} \hspace{2em}  a^{\dag} \ket{1} = \beta \ket{2} \hspace{2em}  a^{\dag} \ket{2} = 0
\end{equation} 
They were able to give trilinear commutation relations for their oscillator. It is worth noting that the authors calculated that the oscillations violating the PEP are suppressed by a factor proportional to $\beta^{2}$ compared to oscillations that do not violate the PEP and vanish for $\beta$ = 0. Following these ideas, Mohapatra and Greenberg (\cite{Greenberg2000}) descibred this model as a modified version of the order-two Green ansatz. They introduced a parameter $\beta$ giving a deformation of Green's trilinear commutators (see \ref{eq:triComm}). For $\beta$ \textrightarrow \ 1 the relations reduce to those of the p = 2 parafermi field. For $\beta$ \textrightarrow \ 0 on the other hand, double occupancy is completely suppressed and fermi theory is obtained. Particles described by this theory were called \textit{parons}. A state of two paronic electrons has the probability to be in a double occupancy state of \betatwo. It was shown by A. Govorkov in \cite{Govorkov1989} that every alteration of Green's commutation relation (like the one discussed here) necessarily has states with negative squared norms. Thus the model of Igantiev and Kuzmin cannot be extended to become a true field theory.

\subsection{Quons}

The idea of a class of particles called \textit{quons} was described by O. W. Greenberg \cite{Greenberg1991}.  The commutator algebra of quons can be obtained as the convex sum of the fermi and bose commutator algebras
\begin{equation}
\label{eq:qmutator1}
 \frac{1+q}{2}[a_{X},a_{Y}^{\dag}]+\frac{1-q}{2}\{a_{X},a_{Y}^{\dag}\} = \delta_{X,Y}
\end{equation} 
or 
\begin{equation}
\label{eq:qmutator2}
 a_{X}a_{Y}^{\dag} - q a_{Y}^{\dag} a_{X} = \delta_{X,Y}
\end{equation} 
In equations \ref{eq:qmutator1} and \ref{eq:qmutator2} the parameter $q$ was introduced, which interpolates between a fermionic ($q$ = -1) and a bosonic ($q$ = 1) commutation relation. For the quonic states to have positive squared norms, this parameter needs to be within -1$\leq q \leq$ 1. For $q$ deviating from $\pm$ 1, the multidimensional representations of the permutation group, which correspond to Young diagrams with more than one row/column, smoothly become more heavily weighted and have a non-zero probability of being realised. That means for a state with two particles, for which only completely symmetric and antisymmetric wave functions are possible, a density matrix \footnote{A density matrix describes a statistical ensemble of several quantum states. This is in contrast to a quantum mechanical mixture of a pure state, described by a state vector.} can be given to represent the mixture of the possible states in the form:
\begin{equation}
 \rho = \frac{1+q}{2}\ket{\phi_{s}}\bra{\phi_{s}} + \frac{1-q}{2}\ket{\phi_{a}}\bra{\phi_{a}}
\end{equation} 
For fermionic quons, the factor $q$ would be close to and slightly larger than -1. For bosonic quons, the factor $q$ would be close to and slightly smaller than 1. When the theory of quonic fermions is related to paronic fermions, where the probability of a state with double occupancy is \betatwo, it follows that
\begin{equation}
 \frac{\beta^{2}}{2} = \frac{1+q_{F}}{2} \Rightarrow q_{F} = \beta^{2} - 1
\end{equation} 
Quonic particles clearly violate the Spin-Statistics connection. It is worth noting nevertheless, that several properties of relativistic theories do hold, like the CPT theorem for example. But as the Spin-Statistics connection does hold for relativistic theories with the usual properties, some property has to fail. This is the property of locality. It turns out that in this framework, observables separated by spacelike separation \footnote{Points with spacelike separation are not connected by a lightcone and are therefore not causally connected.}, do not commute.


\section{Tests of the Pauli Exclusion Principle}

%why test -> l okun, greenberg, jackson, 
%Elliott -> division of experiments; other tests than VIP2

\subsection{Remarks on Testing the Pauli Exclusion Principle}

The question might be asked as to why one should test the PEP and thereby the Spin-Statistcs connection, if it can not be violated in a relativistic theory with the usual properties. O. W. Greenberg gives in \cite{Greenberg2000} several ``external motivations'' which could lead to a possible violation of Spin-Statistics, namely:
\begin{itemize}
 \item violation of CPT
 \item violation of locality
 \item violation of Lorentz invariance
 \item extra space dimensions
 \item discrete space and/or time
 \item noncommutative spacetime
\end{itemize}
As these items are subjected to active research, it seems plausible also to experimentally test the Spin-Statistics connection. Unfortunately, the level of a possible violation, if it is occurring, is unknown. Also there is no reason, why a possible violation should be as small as experiments find it to be. Another important point when testing the PEP is that one does not search for fermions which are ``a bit'' different. If this kind of slightly different fermions would exist, the lowest order pair production cross section would double \cite{Greenberg2000}. This is clearly ruled out by experiments. Because of the indistinguishability of identical particles, all fermions should have the same possibility for an admixture of a symmetric state. This is reflected in the use of the density matrix for the description of states in the case of quons.

\subsection{Experiments for Testing the Pauli Exclusion Principle}
\label{sec:experiments}

According to S. R. Elliott, the various experiments testing the Pauli Exclusion Principle can be grouped into threes classes \cite{Elliott2012}, with respect to the kind of fermionic interaction they are investigating:
\begin{itemize}
 \item Type 1: interactions between a system of fermions and a fermion that has not yet interacted with any other fermion
 \item Type 2: interactions between a system of fermions and a fermion that has not yet interacted with this given system
 \item Type 3: interactions between a system of fermions and a fermion within this system
\end{itemize}
These distinctions between different interactions are necessary due to the MG superselection rule (see also section \ref{sec:IndSymPostSS}). This rule forbids changes of the permutation symmetry of a quantum state in a system where the number of particles is constant. The important difference among these classes is that in a type 3 interaction, the superselection rule forbids a change in permutation symmetry as the investigated fermion already has a defined permutation symmetry with the surrounding system and the number of particles in the system does not change. Subsequently the different types of experiments will be discussed and examples will be given:

\subsubsection{Type 1 experiments:}

The typical experiment of this type uses recently created fermions and lets them interact with the system under investigation. In 1948 Goldhaber and Scharff-Goldhaber \cite{Goldhaber1948} produced electrons with a $\beta$ source and let them capture on Pb atoms. The authors idea was that if the electrons from the $\beta$ source are not subject to the PEP in the electron shell of the Pb atoms, they could cascade to the ground state and thereby emit photons which would be detected. The lack of these photons was used to set an upper limit on the probability for the violation of the PEP. The fundamental point is that the electrons from the $\beta$ source have not yet interacted with any system and are therefore necessarily new to the electronic system of the Pb atoms. As they are new to this system, they can form new quantum states with the Pb atoms. Forming states with a symmetric admixture is not forbidden by the MG superselection rule in this case. This is also true, if the electronic state of the atom has previously been in a completely antisymmetric state. Other sources of recently produced fermions can also be pair-production processes and nuclear reactions.

\subsubsection{Type 2 experiments:}

In this kind of experiments, fermions are brought to a system to interact with it. These fermions have not recently been created, but they have not previously interacted with the system. The typical experiment is the one conducted by Ramberg and Snow \cite{RAMBERG1990}. In this experiment, electrons were introduced to a Cu conductor via a current. These current electrons have no previous interaction with the atoms in the conductor. Therefore the same arguments apply as for type 1 experiments and the formation of a state with an admixture of symmetric exchange symmetry between the atoms of the conductor and a conduction band electron is not forbidden by the MG superselection rule. 

In case of a symmetric admixture in a quantum state formed between an atom in the conductor and the current electrons, the current electron could cascade to the ground state, thereby emitting photons. The lack of detected photons was used to set an upper limit on the probability for the violation of the PEP. An interesting point is the precise origin of the electrons in the conduction band. In an optimal setup, they are coming from a battery. This would guarantee the newness of the electrons. The drawback is that a high current is hard to maintain in this way. If the power of the current source comes from an AC grid the electrons in the conduction band of the conductor will comprise electrons from the conductor itself and the circuitry connecting it to the power supply.

An interesting idea was put forward by E. Corinaldesi \cite{Corinaldesi1967}, who suggested that the PEP is not a kinematic principle but rather a time-dependent effect of interactions and that newly formed system may undergo PEP violating transitions, whose rate decreases with time. This suggestion could be tested with a type 2 experiment. In \cite{Shimony2006} it was suggested that this hypothesis can be tested by crossing an electron and a Ne$^{+}$ ion beam, and to monitor potential photons from PEP violating transitions.

\subsubsection{Type 3 experiments:}

A type 3 experiment searches for a PEP violating transition in a stable fermionic system where the number of particles is constant. Notably the considered systems need to change their permutation symmetry in order to undergo these transitions. Therefore type 3 experiments violate the MG superselection rule and their outcome can not be compared to type 1 and type 2 experiments. 

Nevertheless, many experiments of this kind have been conducted. Pioneers in this kind of experiment were Reines and Sobel \cite{Reines1974}. They were looking for transitions of L-shell electrons to the already occupied K-shell in iodine atoms. The DAMA/LIBRA experiment conducted an analysis of their data regarding the same process \cite{Bernabei2009}. Nuclear processes were also investigated regarding PEP-violating transitions, for example in \cite{Bellini2010} by the Borexino collaboration. The experimenters were looking for non-Paulian transitions of nucleons from the 1p to the 1s nuclear shell. 

\subsubsection{Anomalous Structures}

Another type of experiment is to look for anomalous nuclear and atomic structures. In \cite{Nolte1991} an experiment is reported where nuclear states with three nucleons in the 1s ground state are searched. Non-Paulian atomic states of Be are explored in \cite{Javorsek2000}. 

Some limits on the probability of a violation of the PEP are summed up in table \ref{tab:limits}.

\begin{center}
\label{tab:limits}
\begin{tabular}{ |c|c|c|c| } 
 \hline
 Process & Type &  \betatwo limit & Reference \\ 
 \hline
 \hline
 anomalous atomic transition & 1 & 3 $\times 10^{-2}$ & \cite{Goldhaber1948} \\
 \hline
 anomalous atomic transition & 2 & 4.7 $\times 10^{-29}$ & \cite{Curceanu2011} \\
 \hline
 anomalous atomic transitions & 3 & 6.5 $\times 10^{-46}$ & \cite{Bernabei2009} \\
 \hline
 anomalous nuclear transitions & 3 & 2.2 $\times 10^{-57}$ & \cite{Bellini2010} \\
 \hline
\end{tabular}
\end{center}

The most stringent limit prior to the VIP2 experiment in a system circumventing the MG superselection rule (type 1 + 2 experiments) is set by the VIP experiment \cite{Curceanu2011}. The experimental method of the VIP and VIP2 experiments will now be described.

\section{The VIP2 experimental method}
\label{sec:VIP2expMethod}

%copper - element of group 11 like silver and gold - e- configuration
%source of electrons = current -> ez way to get electrons many; 
%electrons in a conductor, fermi sphere, interactions; cascading process; 
%new symmetry state

As mentioned in section \ref{sec:experiments}, the change of permutation symmetry of a quantum state is not forbidden by the MG superselection rule, when a fermion, which is new to the studied system (e.g. atom, nucleus), interacts with it. These types of experiments were classified as type 1 and type 2. To the best of our knowledge, the most feasible way to introduce a large number of fermions into a system is by introducing a current into a conductor. The number of electrons introduced in this way is for 1 A $\sim 10^{19}$ per second. While moving through the conductor with a velocity influenced by the applied electric potential, these electrons have a certain probability to interact with the atoms in the conductor. Due to this interaction, the electrons from the conduction band can form a new quantum state with the electrons in the atom. It is important to mention that the electrons in the conduction band did not have a defined symmetry with respect to the atomic electrons before this interaction happens, as they come from the current source outside of the conductor. This formation of a new quantum state is the reason why this kind of experiment does not violate the MG superselection rule. After the formation of the new state, the former current electron has the possibility to have symmetric permutation symmetry with respect to particle exchange with the other electrons in the electron shell if the PEP can be violated. This electron sees all the states occupied by the atomic electrons as empty and can occupy them. Therefore it will cascade down into the 1s ground state of the atom, emitting photons as it loses energy. These photons are collected for some time with and without a current. As there are no new electrons introduced during the measurement without current, there are also no photons expected in this time. This measurement is used to determine the background of the energy spectrum. From the difference between the energy spectra in the energy regions where one expects photons from the PEP violating transitions, one can calculate the probability for a violation of the Pauli Exclusion Principle in an atom, or set upper bounds for this probability.

For the VIP2 experiment the conducting material is copper (Cu). It has the atomic number 29 and is part of the group 11 in the periodic table of elements together with for example silver (atomic number 47) and gold (atomic number 79). A common feature of these elements is that they are good conductors for electrical currents. Copper has a resistivity at room temperature of 1.68 $\times 10^{-8}$ $\Omega m $, silver has 1.59 $\times 10^{-8}$ $\Omega m $ and gold has 2.44 $\times 10^{-8}$ $\Omega m $, making silver the best conductor of them. Due to its low cost, copper was the obvious choice for this experiment. The mentioned elements are good conductors due to the unpaired electron in the outermost s-shell. The electronic configuration for copper for example is [Ar] $3d^{10}4s^{1}$. The Fermi energy \footnote{The Fermi energy is the energy of the highest occupied energy state of a system at a temperature of 0 K.} overlaps the 4s orbital \cite{Hilscher2009}. It is a broad band \footnote{An energy band in a solid is a region of allowed states in a E($\vec{k}$) diagram. Here $E$ is the energy and $\vec{k}$ is the wave vector.} which resembles the dispersion relation of free electrons. At finite temperature the electrons of the 4s orbital can move freely in this band (i.e. change their momentum) and be the carrier of the current. 

The energies of the mentioned cascading process were calculated for copper atoms in \cite{DiMatteo2005} using a self-consistent multiconfiguration Dirac-Fock (MCDF) approach. In this case self-consistent is best explained with the help of figure \ref{fig:SelfCons}.
\begin{figure}[h]
 \centering
 \includegraphics[width=0.8\textwidth]{./Figures/SelfConsistency.png}
 \caption{A self-consistent algorithm for calculating the energy of atomic states.}
 % SelfConsistency.png: 597x320 pixel, 72dpi, 21.06x11.29 cm, bb=0 0 597 320
 \label{fig:SelfCons}
\end{figure}
It means after calculating the potential from a charge density at any step and solving the Schrödinger equation with it, the charge density calculated from the Schrödinger equation needs to be the same as the initial charge density. The term ``multiconfiguration'' comes from the fact that the total wave function is described as a linear combination of configuration state functions, which are related to a specific configuration of electrons. Using the Dirac-Fock approach as opposed to the Hartree-Fock means that relativistic effects are accounted for. The relativistic Breit-Dirac Hamiltonian is used which takes into account all electromagnetic interactions of spin $\frac{1}{2}$ particles including spin-orbit coupling and retardation effects. After self-consistency is achieved, the total energy is corrected by vacuum polarization effects (self energy and vacuum polarization). In the whole procedure the ``no pair'' approximation is applied which explicitly excludes electron positron pairs.

The original code for this calculation, which was later adapted for the use of the VIP2 experiment, was described in \cite{Mallow1978}. It calculates the energies of atomic states in an electronic shell in which all but one electron have antisymmetric exchange symmetry. These states violate the PEP. The working principle of the calculation was described as a three step process in \cite{Curceanu2013}:
\begin{itemize}
 \item Step 1: The functional form of the wave function is selected and defined in terms of certain functions (mostly hydrogen-like wave functions) which are combined with certain parameters.
 \item Step 2: An expression for the total energy is derived in terms of these functions and parameters.
 \item Step 3: The variational principle is applied and equations are derived for the valid solutions that are the functions that leave the total energy stationary. In this step self-consistency is checked.
\end{itemize}
The total wave function must then also obey the Hartree-Fock assumptions, for example that the wave function is antisymmetric with the exception of the one electron which has symmetric exchange symmetry. Furthermore, the total wave function needs to be an eigenfunction of the $L^{2},L_{z},S^{2}$ and $S_{z}$ operators. The results of these calculations for copper are summed up in table \ref{tab:transEne}.

\begin{table}[h]
 \centering
  \begin{tabular}{|B|C|B|B|c|}
  \hline
  Transition & Transition energy - PEP violating (eV) &  Transition energy - normal (eV) & Radiative transition rate ($s^{-1}$) &   Multipole Order\\
  \hline
  \hline
  2p$_{\frac{3}{2}}$ $\rightarrow$ 1s$_{\frac{1}{2}}$ (K$_{\alpha 1}$) & 7748 & 8048 & 2.56 $\times 10^{14}$ & E1 + M2\\
  \hline
  2p$_{\frac{1}{2}}$ $\rightarrow$ 1s$_{\frac{1}{2}}$ (K$_{\alpha 2}$) & 7729 & 8028 & 2.63 $\times 10^{14}$ & E1\\
  \hline
  3p$_{\frac{3}{2}}$ $\rightarrow$ 1s$_{\frac{1}{2}}$ (K$_{\beta 1}$) & 8532 & 8905 & 2.68 $\times 10^{13}$ & E1 + M2\\
  \hline
  \end{tabular}
    \caption{Transition rate and energies for PEP violating transitions in copper calculated with the MCDF algorithm \cite{DiMatteo2005}.}
  \label{tab:transEne}
\end{table}

%\end{table}
It is interesting to note that while for normal transitions the K${\alpha _{1}}$ transitions has twice the intensity of K${\alpha_{2}}$, the corresponding PEP-violating K${\alpha_{2}}$ has a slightly higher intensity. This is why for future calculations the energy value of the PEP forbidden K${\alpha{2}}$ line of 7729 eV will be used. Furthermore, as the transition rate of the K${\beta}$ is lower by one order of magnitude, the primary focus of the analysis will be on the K${\alpha}$ transitions. Due to the angular momentum selection rules \footnote{The conservation of angular momentum demands $|J_{i}-J_{f}| \leq \lambda \leq J_{i}+J_{f}$, where $J_{i,f}$ are the initial and final total angular momenta and $\lambda$ is the photons angular momentum. $\lambda$ = 0,1,2,.. for electric and magnetic monopole, dipole, quadrupole, ... transitions. The change of parity for electric transitions is (-1)$^{\lambda}$ and for magnetic transitions it is (-1)$^{\lambda+1}$, which ensures the conservation of overall parity. The parity of a state is (-1)$^{L}$, so it does not change from 2s to 1s. As for electric dipole transitions the parity needs to change, it is electric dipole forbidden.}, the 2s - 1s transition is forbidden, which also holds true for PEP forbidden transitions.

The difference in the transition energies between the normal K-lines and the PEP forbidden K-lines listed in table \ref{tab:transEne} can be explained with figure \ref{fig:forbTrans}.
\begin{figure}[h]
 \centering
 \includegraphics[width=0.7\textwidth]{./Figures/energy_scheme.png}
 % energy_scheme.png: 1215x350 pixel, 72dpi, 42.86x12.35 cm, bb=0 0 1215 350
 \caption{Scheme of normal 2p to 1s transition (left) and a 2p to 1s transition which is violating the PEP (right).}
 \label{fig:forbTrans}
\end{figure}
On the left side of the figure a normal 2p to 1s transition is shown. In this transition an electron from the 2p shell fills a vacancy in the 1s ground state, thereby losing 8048 eV of energy in the form of a photon. On the right side the corresponding PEP violating transition is shown. The electron undergoing the transition cascades down from the 2p into the 1s shell, but in this case, the 1s shell is occupied with 2 electrons. This is only possible because of the symmetric admixture in the symmetry of the wave function. The two electrons in the 1s ground state shield the core potential more than the one electron in the case of the normal transition. Thereby they reduce the effective nuclear charge, which causes the transition energy to be lower for this transition. In the case of the K${\alpha}$ transition for copper, the difference in energy is around 300 eV.
