
%\begin{abstract}
\textbf{\LARGE{Abstract}}

The Pauli Exclusion Principle (PEP) is a fundamental principle in physics, governing the behavior of fermionic particles. Due to its importance, it needs to be tested as precisely as possible. In a pioneering experiment, Ramberg and Snow supplied an electric current to a Cu target, and searched for PEP violating atomic transitions of ``fresh'' electrons from the current. As these transitions are only expected when the current is on, the difference between the spectra with and without current can be used to set an upper limit for the violation of the PEP. The VIP (VIolation of Pauli Exclusion Principle) experiment could set this upper limit to 4.7 $\times$ 10$^{-29}$ with the described method. The VIP2 experiment wants to improve this limit by upgrading crucial components of the setup.

One crucial component of the VIP2 setup are the Silcion Drift Detectors which detect the possible photons from PEP violating transitions. These detectors were tested in the laboratory of the Stefan Meyer Institute together with argon cooling, which kept their temperature at around 100 K. Their energy and time resolution were assessed during these tests and equaled 150 eV (FWHM) at 6 keV and 380 ns (FWHM) respectively. Another essential component of the setup are the 32 plastic scintillator bars read out by two Silicon Photo Multipliers each. The time resolution of one of these systems was 2.3 ns (FWHM). The system was installed to veto events in the SDDs caused by external ionizing radiation. The detection probability for 500 MeV e$^{-}$ was tested at the beam test facility at LNF (Italy), were 98 \% or these particles were detected. The outcome of all these measurement are fulfilling expectations and suffice for the purpose of the experiment. 

After exhaustive tests, the setup was transported to the underground laboratory of Gran Sasso (LNGS). Data was taken from February 2016 until November 2017. An amount of about 142 days of data without current and 81 days with 100 A current were taken. Comparison of the data to Geant4 simulation showed that the majority of the background is induced by $\gamma$ radiation originating from radioactive isotopes of the rocks of the mountain. The energy resolution around 8 keV, where events from the PEP-violating transition are expected, was measured. For the data without current, it equals 178 eV. For the data with current it is diminished to 189 eV, probably due to electronic noise introduced by the high current. Several analysis techniques for investigating the difference of the two energy spectra, and calculate the probability for a violation of the Pauli Exclusion Principle, were applied. The standard analysis of subtraction of the spectra in a region of interest defined around the expected energy of the forbidden transition yields a new upper limit of 1.87 $\times$ 10$^{-29}$. This is an improvement to the value set by the VIP experiment by a factor of 2.5.
%\end{abstract}