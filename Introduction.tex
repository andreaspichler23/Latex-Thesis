\chapter{Introduction}
\label{chap:Introuction}

%general introduction without going into too much detail; history of experiments in Spin statistics violation, why is it important to test PEP, no superselection rule or measurement principle

The Pauli Exclusion Principle (PEP) is a fundamental principle in physics, valid for identical-fermion systems. It was formulated in 1925 by the austrian physicist Wolfgang Pauli. It states that two fermions (particles with half integer spin) can not occupy the same quantum state simultaneously. Examples for fermionic particles are elemenatry particles such as quarks, leptons (electron, muon and tauon) and neutrinos. Also composite particles can be fermions (e.g. protons and neutrons). Electrons, which make up the electronic shell of atoms, are fermions and therefore obey the PEP. For the case of electronic shells the PEP is equivalent to the statement that two electrons can not have the same principal quantum number \textit{n}, angular momentum quantum number \textit{l}, magnetic quantum number \textit{$m_{l}$} and spin quantum number \textit{$m_{s}$} at the same time. This means that two electrons can share the quantum numbers \textit{n}, \textit{l} and \textit{$m_{l}$}, as long as they have different spin quantum number \textit{$m_{s}$} ($\pm$ \textonehalf).

The PEP forms the basis of the periodic table of elements, as it prevents all electrons in a shell to condense into the ground state. Therefore it is responsible for the occupation of the electronic shells and the chemical properties of elements. Also it is connected to the stability of neutron stars, as the neutron degeneracy pressure, which is caused by the PEP, prevents them from collapsing under their own gravitational pressure. Another phenomenon it is intimately connected to is electric conductivity, which will be described in chapter \ref{chap:Physics}.

Due to the fundamental place of the PEP in quantum field theory, many researches were interested in testing it. In the year 1948, the Pauli Exclusion Principle was tested by Goldhaber and Scharff-Goldhaber \cite{Goldhaber1948}. Their experiment was designed to determine if the particles making up $\beta$-radiation were the same as electrons, but it was later used to test the Pauli Exclusion Principle. The experiment was done by shining electrons the particles from a $\beta$ source onto a block of lead. The authors thought, if these $\beta$ particles were different from electrons, they could be captured by the lead atoms and cascade down to the ground state without being subject to the PEP. The X-rays emitted during this cascading process were used to set an upper bound for the probability that the PEP is violated. Another thorough test was conducted in 1988 by E. Ramberg and G. A. Snow \cite{RAMBERG1990}. They introduced a current into a copper conductor. The electrons of the current then had a chance to be absorbed by the copper atoms and form a new quantum state. The experimenters searched for states having a symmetric component in an otherwise antisymmetric state. These states were identified by X-rays they emitted while cascading to the ground state. The same principle was later employed in the VIP experiment, which was able to set a new upper limit for the probability for the violation of the Pauli Exclusion Principle of:%
%
\begin{equation*}
\frac{\beta^{2}}{2} \leq 4.7 * 10^{-29}                                                                                                                                                                                                                                                                                                                                                                                                                                                                                                                                                                                                                                                                                                                                                                                                                                                                                                                                                                                                                                                                                                                                                                                                                                                                                                                                                                                                                                                                                                      \end{equation*}
%
(\cite{Curceanu2011a}, \cite{Pietreanu2014}).The follow-up experiment VIP2 is currently taking data in the Laboratori Nazionali del Gran Sasso (LNGS). The current results, the setup configuration and various other aspects of this experiment will be discussed in the following chapters.